% !TeX spellcheck = en_US

\chapter{Examples}

\marginpar{Goal of this chapter}This chapter is to demonstrate some of the capabilities of this \LaTeX{} template.
Please take a good look at this chapter and try to follow the guidelines.\\

\section{Equations}
\subsection{some equations}
\marginpar{Equations}\glspl{Equation}\index{Equation} can easily be written using the \emph{\gls{Equation}}
environment. Inline equations are inserted with $\sqrt{-1} = i $. Equations can
also be labeled, so it is possible to reference them. This should be done for
all important equations. 

\begin{equation}
	x^2 + y^2 = 1
	\label{equ:bsp_chapter:example_equation}
\end{equation}

There are several environments for multi line equations. A very useful one
is \emph{align}, see equation \ref{equ:bsp_chapter:example_align}.

\begin{align}
	\oint \vec{E} \cdot d \vec{A} &= \frac{q}{\epsilon_0} \\
	\oint \vec{B} \cdot d \vec{A} &= 0 \\
	\oint \vec{E} \cdot d \vec{s} &= - \frac{d \Phi_B}{dt}
	\label{equ:bsp_chapter:example_align}
\end{align}

\marginpar{Figures and Tables}Images\index{Image} are always inserted inside a \emph{figure}
environment. If possible, it is advisable to use \texttt{[tb]} as position.
Always remember to add a caption and a label, so you can reference the image
like this: \ref{fig:bsp_chapter:example_figure}. If possible, images should be
inserted as vector graphics, e.g. \texttt{eps} or \texttt{pdf} - or even drawn
manually in TikZ. \\ 

\begin{figure}[t]
	\centering
	\includegraphics[width=2cm]{chapters/bsp_chapter/images/thumbs_up.jpg}
	\caption{An example image}
	\label{fig:bsp_chapter:example_figure}
\end{figure}

Tables\index{Table} can be used quite similarly. They are inserted inside a \emph{table}\index{Table!tabular}
environment, as shown in table \ref{tab:bsp_chapter:example_table}. \\

\begin{table}[t]
	\centering
	\begin{tabular}{lcrp{4cm}} \toprule
		some & text & is shown & here \\ \midrule
		there is more & text here & and here & cool. \\ \bottomrule
	\end{tabular}
	\caption{A sample table}
	\label{tab:bsp_chapter:example_table}
\end{table}

Another useful tool is the \emph{tabularx} environment. \index{Table!tabularx}
It lets the user specify the total width of the table, instead of each column. 
An example is shown in table \ref{tab:bsp_chapter:example_tabularx}.

Please read the documentation of the
\emph{booktabs}\footnote{\url{http://www.ctan.org/pkg/booktabs}} 
package to find information on how to create good tables.
Always remember the first two guidelines and try also to stick to the other three:
\begin{enumerate}
	\item Never, ever use vertical rules.
	\item Never use double rules.
	\item Put the units in the column heading (not in the body of the table).
	\item Always precede a decimal point by a digit; thus $0.1$ \emph{not} just $.1$.
	\item Do not use \enquote{ditto} signs to repeate a value. In many circumstances a blank will serve just as well. If it won't, then repeat the value.
\end{enumerate}

\begin{table}[t]
	\centering
	\begin{tabularx}{0.9\linewidth}{lXX} \toprule
		some & text & is shown here \\ \midrule
		there is more & text here. & cool. \\ \bottomrule
	\end{tabularx}
	\caption{Tabularx example}
	\label{tab:bsp_chapter:example_tabularx}
\end{table}

\marginpar{Quotes}Quotes can easily be made using the \enquote{csquotes} package.
Citing text passages is easily done: \textquote[me][!]{First argument: citation,
second argument: terminal punctuation} Whole block quotes are also easily 
possible.

\blockquote{Formal requirements in academic writing frequently demand that
quotations be embedded in the text if they are short but set oV as a distinct
and typically indented paragraph, a so-called block quotation, if they are
longer than a certain number of lines or words. In the latter case no quotation
marks are inserted.}

\marginpar{Bibliography}The bibliography is created using \emph{Bibtex}. The
standard format is set to \texttt{ieeetr}, which is the IEEE Standard. There
are example entries for different types of  in the separate bibliography file
\cite{article} \cite{book} \cite{booklet} \cite{conference} \cite{inbook}
\cite{incollection} \cite{manual} \cite{mastersthesis} \cite{misc}
\cite{phdthesis} \cite{proceedings} \cite{techreport} \cite{unpublished}.

\marginpar{Index \& Glossary}All glossary entries are made in the sepearte file \texttt{glossary.tex}.
They can then be used with \gls{Equation}.
Acronyms are defined as shown there and used similarly. 
The first time, it will be \emph{\gls{svm}}.
The second time: \emph{\gls{svm}}.
The glossary has to be created manually by invoking \texttt{makeindex -s doku.ist -t doku.glg -o doku.gls doku.glo}.
The index is simply created by using \texttt{index\{text\}}. It is generated automatically.